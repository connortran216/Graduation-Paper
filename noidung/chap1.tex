\chapter{Introduction}
\section{Reason For Choosing Topic}
\label{section:reason_topic}
\noindent
	
	Ultrasound images are the most widely used in the modern medical field to monitor and diagnose a fetus during pregnancy. Many women prefer to use it to follow the development of the fetus because it is economical, real-time monitoring, and extremely safe compared to X-rays or other types of imaging systems that use radiation based-method \cite{rueda}, \cite{shahin}. Ultrasound examinations provide parents with a valuable opportunity to understand the health status of their unborn child. Although ultrasound is considered to be efficient and safe, it still has several disadvantages.
	
	The process of ultrasound imaging depends mostly on the one who operates the system for its signature noises on images such as shadows and reverberations, which makes the image extremely hard to observe and diagnose the fetus’ status \cite{shahin}. That is the reason why ultrasound systems can only be used by well-trained, professional specialists, doctors, and sonographers, which leads to the shortage of human resources in poor and developing countries.
	
\section{Target Implementation}
\label{section:target_implementation}
\noindent
	
	In the fetal ultrasound image, ultrasound allow the visualization of some body features, possibly other parts such as fingers and toes of the fetus. Based on these important features, many biometric measurements are applied by doctors or ultrasound imaging operators. For example, the crow-rump length (CRL) and head circumference (HC) are 1 commonly calculated to estimate the gestational age (GA) and give diagnosis about growth of the fetus. The CRL widely uses for its accuracy to estimate the GA of the fetus in the age between 8 weeks 4 days and 12 weeks. After 13 weeks, specialists usually use HC for its accuracy instead of CRL. The instruction of HC measurement state that HC should be measured in a transverse section of the head with a central midline echo, interrupted in the anterior third by the cavity of the septum pellucidum with the anterior and posterior horns of the lateral ventricles in view \cite{thomas}. 
	
	The HC measurement steps are proceeded manually which make the result unstable when being conducted by different doctors. The idea of creating an automated system is born base on the above obstacle. With the support from computer, the HC measurement result will be no longer affected by observer variability along with the shortage of sonographers.

    
    