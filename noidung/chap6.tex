\chapter{Kết luận và hướng phát triển}
\noindent

Từ quá trình tìm hiểu và kết quả thực nghiệm đi đến kết luận và xác định hướng phát triển cho đề tài trong tương lai.

\section{Kết quả đạt được}
	
	
%Phương pháp được đề xuất trong bài là mô hình Weakly-supervised hierarchical text classification. Mô hình được thiết kế để phân loại văn bản có tính phân cấp và dữ liệu không cân bằng. Bằng cách tạo ra dữ liệu giả cho tiền huấn luyện mô hình từ các từ khóa người dùng cung cấp hoặc từ một số tài liệu mẫu từ tập đầu vào được đánh nhãn chính xác. 

%Mô hình này sẽ được áp dụng cho tập tài liệu tiếng Việt với nhiều trường hợp khác nhau như tập văn bản dài, tập ngắn, tập chứa nhiều lớp hiếm, \ldots. Với mô hình này ta sẽ cần ít nhân lực hơn trong việc gán nhãn ban đầu nhưng lại cần đến nguồn nhân lực có trình độ chuyên môn đảm nhận việc gán nhãn. Nhờ khả năng tạo ra dữ liệu giả dựa trên các dữ liệu thật đã giúp giảm bớt chi phí trong quá trình thu thập dữ liệu và chất lượng dữ liệu sẽ tốt hơn.