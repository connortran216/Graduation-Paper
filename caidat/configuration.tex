%\setcounter{secnumdepth}{1}% number \part and \chapter in book classes
%\addtocounter{tocdepth}{-1}% “delete” lowest level from TOC 
\mathtoolsset{showonlyrefs}




% correct bad hyphenation here
\hyphenation{op-tical net-works semi-conduc-tor}

% define language python
\lstset{
	language=Python,
	tabsize=3,
	escapeinside={@*}{*@},
	%frame=lines,
	label=code:sample,
	frame=shadowbox,
	rulesepcolor=\color{gray},
	xleftmargin=20pt,
	framexleftmargin=15pt,
	keywordstyle=\color{blue}\bf,
	commentstyle=\color{gray},
	stringstyle=\color{red},
	numbers=left,
	numberstyle=\tiny,
	numbersep=5pt,
	breaklines=true,
	showstringspaces=false,
	basicstyle=\footnotesize,
	emph={food,name,price},emphstyle={\color{magenta}}
}

\renewcommand{\baselinestretch}{1.5} 
\usepackage[left=3.50cm, right=2.00cm, top=3.5cm, bottom=3.00cm]{geometry}


% formatting taken from http://tex.stackexchange.com/questions/149708/simple-list-of-abbreviations-manually
% sorting taken from http://www.latex-community.org/forum/viewtopic.php?f=44&t=16419

\usepackage{datatool}

% Define a convenient command to add a line
% to the database
\newcommand*{\addacronym}[2]{%
	\DTLnewrow{acronyms}%
	\DTLnewdbentry{acronyms}{Acronym}{#1}%
	\DTLnewdbentry{acronyms}{Description}{#2}%
}
% formatting
\newcommand{\tocfill}{\cleaders\hbox{$\m@th \mkern\@dotsep mu . \mkern\@dotsep mu$}\hfill}
\newcommand{\abbrlabel}[1]{\makebox[3cm][l]{\textbf{#1}\ \tocfill}}
\newenvironment{abbreviations}{\begin{list}{}{\renewcommand{\makelabel}{\abbrlabel}%
\setlength{\labelwidth}{3cm}\setlength{\leftmargin}{\labelwidth+\labelsep}%
\setlength{\itemsep}{0pt}}}{\end{list}}





